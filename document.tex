% Credits to https://tex.stackexchange.com/questions/8827/preparing-cheat-sheets
\documentclass[10pt]{article}
\usepackage{multicol}
\usepackage{calc}
\usepackage{ifthen}
\usepackage[portrait]{geometry}
\usepackage{amsmath,amsthm,amsfonts,amssymb}
\usepackage{color,graphicx,overpic}
\usepackage{hyperref}


\pdfinfo{
  /Title (example.pdf)
  /Creator (TeX)
  /Producer (pdfTeX 1.40.0)
  /Author (Seamus)
  /Subject (Example)
  /Keywords (pdflatex, latex,pdftex,tex)}

% This sets page margins to .5 inch if using letter paper, and to 1cm
% if using A4 paper. (This probably isn't strictly necessary.)
% If using another size paper, use default 1cm margins.
\ifthenelse{\lengthtest { \paperwidth = 11in}}
    { \geometry{top=.5in,left=.5in,right=.5in,bottom=.5in} }
    {\ifthenelse{ \lengthtest{ \paperwidth = 297mm}}
        {\geometry{top=1cm,left=1cm,right=1cm,bottom=1cm} }
        {\geometry{top=1cm,left=1cm,right=1cm,bottom=1cm} }
    }

% Turn off header and footer
\pagestyle{empty}

% Redefine section commands to use less space
\makeatletter
\renewcommand{\section}{\@startsection{section}{1}{0mm}%
                                {-1ex plus -.5ex minus -.2ex}%
                                {0.5ex plus .2ex}%x
                                {\normalfont\large\bfseries}}
\renewcommand{\subsection}{\@startsection{subsection}{2}{0mm}%
                                {-1explus -.5ex minus -.2ex}%
                                {0.5ex plus .2ex}%
                                {\normalfont\normalsize\bfseries}}
\renewcommand{\subsubsection}{\@startsection{subsubsection}{3}{0mm}%
                                {-1ex plus -.5ex minus -.2ex}%
                                {1ex plus .2ex}%
                                {\normalfont\small\bfseries}}
\makeatother

% Define BibTeX command
\def\BibTeX{{\rm B\kern-.05em{\sc i\kern-.025em b}\kern-.08em
    T\kern-.1667em\lower.7ex\hbox{E}\kern-.125emX}}

% Don't print section numbers
\setcounter{secnumdepth}{0}


\setlength{\parindent}{0pt}
\setlength{\parskip}{0pt plus 0.5ex}

% My Environments
% Tab alike command
\newcommand\tab[1][1cm]{\hspace*{#1}}

% -----------------------------------------------------------------------

\begin{document}
\raggedright
\footnotesize
\begin{multicols}{2}


% multicol parameters
% These lengths are set only within the two main columns
%\setlength{\columnseprule}{0.25pt}
\setlength{\premulticols}{1pt}
\setlength{\postmulticols}{1pt}
\setlength{\multicolsep}{1pt}
\setlength{\columnsep}{2pt}

\begin{center}
     \Large{\underline{Swedish Grammar}} \\
\end{center}

% Some Swedish characters
% å  => \aa
% Å => \AA
% ä  => \"a
% Ä => \"A
% ö  => \"o

\section{Nouns}
	[Common]\textbf{En} pojke\newline
	[Neuter]\textbf{Ett} \"apple\newline
	\par
	[Morpheme order]\underline{Noun stem}(plural)(definite)(genitive -s)
\subsection{Plurals}
\subsubsection{[Indefinite]Plurals}
	Five declensions -or, -ar, -(e)r, -n, -\emph{null}\newline
	1. [Common]En flick\textbf{a}, flera flick\textbf{or}\newline
	2. [Common]En hund, flera hund\textbf{ar}\newline
	3. En bok, flera b\"ok\textbf{er}\newline
	4. [Neuter, end in vowel]Ett \"apple, flera \"apple\textbf{n}\newline
	5. Ett djur, flera djur\newline
	\par
	\emph{Some exceptions exist!}
\subsubsection{[Definite]Plurals}
	Five declensions -or, -ar, -(e)r, -n, -\newline
	1. Flaska\textbf{n}, flask\textbf{orna}\newline
	2. [Common]Stol\textbf{en}, stol\textbf{arna}\newline
	3. [Mostly common]Sak\textbf{en}, sak\textbf{erna}\newline
	4. [Neuter, ends in vowl]Hj\"arta\textbf{t}, hj\"arta\textbf{na}\newline
	5. [Mostly neuter]Horn\textbf{et}, horn\textbf{en}
\section{Pronouns}
\subsection{[Personal]Pronouns}
	\subsubsection{Nominative|Objective|Possesive(common,neutre,pleural)}
	1. Jag|mig|min,mitt,mina\newline
	2. Du|dig|din,ditt,dina\newline
	3a. [Masculine]Han|honom|hans\newline
	3b. [Feminine]Hon|henne|hennes\newline
	3c. [Common]Den|den|dess\newline
	3d. [Neuter]Det|det|dess\newline
	3e. [Indefinite ("one")]Man|en|ens\newline
	3f. [Reflexive]\emph{null}|sig|sin,sitt,sina\newline
	4. Vi|oss|v\aa r,v\aa rt,v\aa ra\newline
	5. Ni|er|er,ert,era (ers(honorific))\newline
	6a. De|dem|dera\newline
	6b. [Reflexive]\emph{null}|sig,sitt,sina\newline
\section{Present tense}
	1. Hopp\textbf{ar} [regular]\newline
	2. Sov\textbf{er} [regular,strong]\newline
	3. Bo\textbf{r} [regular,strong]\newline
\section{Questions}
	1. Vad [what]\newline
	2. Var [where (location)]\newline
	3. Vart [where (direction)]\newline
	4. Hur [how]\newline
	5. Varf\"or [why]\newline
	6. Vem|vems [who|whose]\newline
	7. N\aa r [when]\newline
	8. Vilken|vilket|vilka [which]\newline
	9. Vilka [who (only for plurals)]\newline
\section{Prepositions}
	1. P\aa [on|under|to|from]\newline
	2. under[under]\newline
	3. \textbf{TBD}\newline
\section{Conjunctions}
\subsection{Normal conjunctions}
	1. Och [and]\newline
	2. Men|utan [but]\newline
	3. Eller [or]\newline
\subsection{Subordinate conjunctions}
	1. Att [that]\newline
	2. Eftersom [because]\newline
	3. D\"arf\"or att [because (never in beginning of sentence)]\newline
	4. Innan [before]\newline
	5. Framf\"or [in front of]\newline
	6. Efter [after]\newline
	7. Utanf\"or [outside]\newline
	8. Bredvit [next to]\newline
	9. Hoss oss [at our place]\newline
	10. Hos [at]\newline
	11. Mot [towards|against]\newline
	12. Fram\aa t [ahead|forward]\newline
	13. F\"ore [before]\newline
	14. Bakom [behind]\newline
	15. F\"ore [before]\newline
	16. Medan [while]\newline


\end{multicols}
\end{document}
